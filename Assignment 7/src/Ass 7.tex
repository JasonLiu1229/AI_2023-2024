% Just ignore everything between this and the next commented line!

\documentclass[12pt]{article}
 \usepackage[margin=1in]{geometry}
 \usepackage[usenames,dvipsnames]{xcolor}
\usepackage{amsmath,amsthm,amssymb,amsfonts,
hyperref, color, graphicx,ulem}
\usepackage{datetime}
\usepackage{biblatex}
\newcommand{\N}{\mathbb{N}}
\newcommand{\Z}{\mathbb{Z}}
\newcommand{\Q}{\mathbb{Q}}
\newcommand{\mm}{\textcolor{blue}{You need to use math mode whenever you are writing logic symbols, variables, sets etc. and you need to use text whenever you are writing words. If you are not sure what this means, please speak to me.}}
\newcommand{\al}{\textcolor{blue}{You might try the align environment as shown below:
\begin{align*}
y=&a+b+a\\
=&2a+b
\end{align*}
The \& symbol allows you to line up the ='s or anything else you want to line up! It also saves you the space between lines of display style equation!}}
\newcommand{\steps}{\textcolor{blue}{You need to use the claim environment to give a claim, then close that and use the proof environment to give your proof.}}
\newcommand{\nproof}{\textcolor{blue}{I see what you are trying to say, but this is not a proof. Please see me if you do not understand what I mean by this.}}
\newcommand{\equal}{\textcolor{blue}{You can only use ``='' between two things which are actually equal. This is not just a way of stringing things together!}}
\newcommand{\mul}{\textcolor{blue}{This is not a valid way of denoting multiplication. You can use $\times$ or $\cdot$, or often the implied multiplication of adjacent variables.}}
\newcommand{\ex}{\textcolor{blue}{You cannot just give one example. You are tasked with showing that this claim holds for all possible values.}}
\newcommand{\thus}{\textcolor{blue}{``therefore'', ``thus", and other words of this flavor should be used only when the preceding sentence leads us to the following sentence. It is not just a way to string things together!}}
\newcommand{\words}{\textcolor{blue}{You need punctuation and words. Remember a proof is supposed to walk the reader through your thought process.}}
\newcommand{\order}{\textcolor{blue}{As a general rule, if the sentences can be reorganized and the proof doesn't make significantly less sense, then it is probably not very well structured! The arguments should flow into each other. Generally sentences should justify each other and you should feel like you are building one thought on top of another.}}
\newcommand{\pr}{\textcolor{blue}{Proofread!}}
\newcommand{\sen}{\textcolor{blue}{You can't start a sentence with a symbol.}}
\newcommand\score[1]{\textcolor{blue}{\textbf{ (score: #1) }}}
\newcommand\blue[1]{\textcolor{blue}{#1}}
\let\div\undefined
\DeclareMathOperator{\div}{div}
\DeclareMathOperator{\dom}{dom}
\DeclareMathOperator{\im}{im}
\newenvironment{Question}[2][Question]{\begin{trivlist}
\item[\hskip \labelsep {\bfseries #1}\hskip \labelsep {\bfseries #2.}]}{\end{trivlist}}
\newenvironment{claim}[2][Claim]{\begin{trivlist}
\item[\hskip \labelsep {\bfseries #1}\hskip \labelsep {\bfseries #2.}]}{\end{trivlist}}

% You can ignore all the code written above. Most of it is so I can make common comments on your work easily and quickly.

\begin{document}

\title{Artificial Intelligence}
\author{Jason}
\date{\today; \currenttime}
% Add your name in above
\maketitle

\begin{center}\begin{LARGE} Assignment 0 \end{LARGE}\end{center}

\begin{Question}{1}
 Exact Inference Observation: Write down the equation of the inference problem you are trying to solve.
\end{Question}
%\text{{new\_belief}}[p] = \text{{getObservationProb}}(\text{{observation}}, \text{{gameState.getPacmanPosition()}}, p, \text{{self.getJailPosition()}}) \times \text{{old\_belief}}[p]
new\_belief[p] = $getObservationProb(observation, gameState.getPacmanPosition(), p, $\\
$self.getJailPosition()) * old\_belief[p]$\\
getObservationProb(observation, pacmanPosition, position, jailPosition) = \\ P(noisydistance \textbar pacman position, ghost position)
\begin{Question}{2}
 Exact Inference with Time Elapse: Write down the equation of the inference problem you are trying to solve.
\end{Question}
new\_belief[p1] = $\sum_{\substack{p \in allPositions \\ old\_belief[p] \neq 0}} old\_belief[p] \cdot getPositionDistribution(gameState, p)[p1]$\\
getPositionDistribution(gameState, ghostPosition) = P(ghost position at time t+1 \textbar ghost position at time t)
\begin{Question}{3}
 Alternative answer for GreedyBusterAgent
\end{Question}
Correct answer for GreedyBusterAgent: \\
Average score: 754.7\\
Alternative answer for GreedyBusterAgent: \\
Average score: 758.7\\
Alternative answer 2 for GreedyBusterAgent: \\
Average score: 754.3\\
The first alternative answer is better than the correct answer and there is also a variable k that could be played with to change the score.\\
The first alternative answer is still a bit greedy but added with some randomness so most.\\
The second alternative answer is a bit less greedy, the strategy is considered less greedy because it doesn't always choose the action that brings Pacman closest to the nearest ghost.
Instead, it uses a probability distribution to decide which ghost to chase.
This means that while the closest ghost has the highest probability of being chosen, there's still a chance that a farther ghost might be selected.
This introduces an element of randomness and exploration into the decision-making process, which can potentially lead to better overall strategies.
\begin{Question}{4}
 Joint Particle Filter Time Elapse and Full Test: As you run the autograder note that q14/1-JointParticlePredict and q14/2-JointParticlePredict test your elapseTime implementations only, and q14/3-JointParticleFulltests both your observeUpdate and elapseTime implementations.
 Notice the dif-ference between test 1 and test 3.
 In both tests, pacman knows that the ghostswill move to the sides of the gameboard.
 What is different between the tests,and why?
\end{Question}
In the first test we predict the ghost position, so the longer the ghost stays in a certain position, the more likely it is to be in that position, so pacman will assume more and more that the ghost is in that position.\\
For the second test, we update based on the last elapse time.
So we look more to recent estimates of the ghost position, instead of looking at the whole history of the ghost position.\\
The second test is more accurate because it is based on the most recent information, while the first test is based on the whole history of the ghost position, making use of predictions.
\end{document}